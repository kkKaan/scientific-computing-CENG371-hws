\documentclass[11pt,a4paper, margin=1in]{article}
\usepackage{fullpage}
\usepackage{amsfonts, amsmath, pifont}
\usepackage{amsthm}
\usepackage{graphicx}
\usepackage{float}

\usepackage{tkz-euclide}
\usepackage{tikz}
\usepackage{pgfplots}
\pgfplotsset{compat=1.13}

\usepackage{geometry}
 \geometry{
 a4paper,
 total={210mm,297mm},
 left=10mm,
 right=10mm,
 top=10mm,
 bottom=20mm,
 }

 \author{
  Karaçanta, Kaan\\
  \texttt{e244854@metu.edu.tr}
}

\newcommand{\mySin}[1]{\textstyle\sin\left(#1\right)}
\newcommand{\myCos}[1]{\textstyle\cos\left(#1\right)}
\usepackage{hyperref}

\usepackage{inconsolata}
\usepackage{listings}
\usepackage{xcolor}
\usepackage[utf8]{inputenc}
\usepackage[T1]{fontenc}

\definecolor{codegreen}{rgb}{0,0.6,0}
\definecolor{codegray}{rgb}{0.5,0.5,0.5}
\definecolor{codepurple}{rgb}{0.58,0,0.82}
\definecolor{backcolour}{rgb}{0.95,0.95,0.92}

\lstdefinestyle{mystyle}{
    backgroundcolor=\color{backcolour},
    commentstyle=\color{codegreen},
    keywordstyle=\color{magenta},
    numberstyle=\tiny\color{codegray},
    stringstyle=\color{codepurple},
    basicstyle=\ttfamily\footnotesize,
    breakatwhitespace=false,
    breaklines=true,
    captionpos=b,
    keepspaces=true,
    numbers=left,
    numbersep=5pt,
    showspaces=false,
    showstringspaces=false,
    showtabs=false,
    tabsize=2
}

\lstset{style=myStyle}

\title{CENG 371 - Scientific Computing \\
Fall' 2024 - 2025 \\
Homework 2}

\date{}

\begin{document}
\maketitle

\noindent\rule{19cm}{1.2pt}


\section*{Question 1}

\begin{enumerate}
    % \item % Implement the power method. (Signature: [eVal, eVec] = power_method(A, V); where A is the matrix and V is an optional starting vector.)
    % \item % Implement the shifted inverse power method. (Signature: [[eVal, eVec] = inverse_power(A, alpha), where A is the matrix alpha is the shift value. eVal, eVec is the eigenvalue/vector that is closest to alpha)
    \setcounter{enumi}{2}
    \item % Find the largest and smallest (in magnitude) eigenvalues and the corresponding eigenvectors of matrix A = [2 -1 0 0 0 ; -1 2 -1 0 0 ; 0 -1 2 -1 0 ; 0 0 -1 2 -1 ; 0 0 0 -1 2]
    
    \item % Find the largest eigenvalue eigenvector pair by hand of matrix B. Where B = [0.2 0.3 -0.5 ; 0.6 -0.8 0.2 ; -1 0.1 0.9] (You can use the identity Av = λv). Do the same using the power method. Use starting vector v where v = [1, 1, 1]T . Reflect on your findings.

\end{enumerate}


\section*{Question 2}

\begin{enumerate}
    
    \item
    
    \item
    
    \item
    
    \item

\end{enumerate}


\end{document}